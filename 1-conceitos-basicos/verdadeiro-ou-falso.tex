% !TeX encoding = UTF-8
% !TeX spellcheck = pt_BR
% !TeX root = ../template.tex
% !Tex tags = 

\question[10]

Use (\textbf{V}) para verdadeiro e (\textbf{F}) para falso nas seguintes sentenças.

\begin{parts}
	
	\part \tf[F] A função \emph{printf} sempre começa a imprimir no início de uma nova linha.
	
	\part \tf[T] Todas as variáveis precisam ser declaradas antes de serem usadas.
	
	\part \tf[T] Uma variável do tipo \emph{float} pode guardar números reais.
	
	\part \tf[T] O operador \textbf{++} aumenta o valor da variável em $1$ enquanto o operador \textbf{\textendash\textendash} diminui o valor em 1.
	
	\part \tf[F] O operador \emph{=} é usado para comparação enquanto o operador \emph{==} é usado para atribuir um valor.

		\part \tf[T] Espaço em brancos podem ser inseridos entre duas palavras para melhorar a legibilidade do código


	\part \tf[T] Palavras-chaves não podem serem usadas como nomes de variáveis.
		
\end{parts}
