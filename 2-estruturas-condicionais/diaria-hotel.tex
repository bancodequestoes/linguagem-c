% !TeX encoding = UTF-8
% !TeX spellcheck = pt_BR
% !TeX root = ../avaliacao.tex

\question[10]

Um hotel cobra R\$ 60.00 a diária e mais uma taxa de serviços. A taxa de serviços é de:

\begin{parts}
	\part R\$ 5.50 por diária, se o número de diárias for maior que 15;
    \part R\$ 6.00 por diária, se o número de diárias for igual a 15;
	\part R\$ 8.00 por diária, se o número de diárias for menor que 15.
\end{parts}

Construa um algoritmo que, dado o número de diárias, mostre o total da conta de um cliente

\begin{solution}
	\begin{lstlisting}
#include <stdio.h>

int main(){

	int diarias;
	float valorTotal;
	
	printf("Digite o numero de diarias: ");
	scanf("%d", &diarias);
	
	if(diarias < 15){
		valorTotal = 60.0*diarias + 8.0*diarias;
	}
	if(diarias == 15){
		valorTotal = 60.0*diarias + 6.0*diarias;
	}
	if(diarias > 15){
		valorTotal = 60.0*diarias + 5.5*diarias;
	}
	
	printf("Valor Total: %f", valorTotal);
}
\end{lstlisting}
\end{solution}
