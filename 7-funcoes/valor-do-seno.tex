% !TeX encoding = UTF-8
% !TeX spellcheck = pt_BR
% !TeX root = ../template.tex
% !Tex tags = 

\question[10]

Faça uma função que receba como parâmetro o valor de um ângulo em graus e calcule o valor do seno desse ângulo usando sua respectiva série de Taylor:

\begin{equation*}
	\text{sen } x = \sum_{n=0}^{\infty} \frac{(-1)^n}{(2n+1)!}x^{2n+1} = x - \frac{x^3}{3!} + \frac{x^5}{5!} - \ldots 
\end{equation*}

para todo $x$, onde $x$  é o valor do ângulo em radianos. Considerar $\pi = 3.141593$ e $n$ variando de 0 até 5.