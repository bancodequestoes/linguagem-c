% !TeX encoding = UTF-8
% !TeX spellcheck = pt_BR
% !TeX root = ../template.tex
% !Tex tags = 

\question[10]

Preencha os espaços nas seguintes sentenças:

\begin{parts}
	
	\part O comando \fillin[return], em uma função é usado para passar o valor de uma expressão ou um valor fixo de volta para a função chamadora.
	
	\part Quando uma função não retorna valor, o tipo do retorno da função deve ser \fillin[void].

	\part A função \fillin[main] é obrigatória em todo código em C.
	
	\part Os  \fillin[parâmetros] são informações necessárias que a função precisa para ser executada.

	\part O tipo do retorno deve ser \fillin[float] se a função tem \emph{return 2.2;}.

	\part Todo programa em C começa pela função \fillin[main].
	
	\part Uma função que chama a si mesma, direta ou indiretamente, é uma função \fillin[recursiva]. 
	
	\part A função  \fillin[rand] é usada para produzir números aleatórios.
	
\end{parts}
