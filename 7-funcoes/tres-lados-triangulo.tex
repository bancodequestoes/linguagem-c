% !TeX encoding = UTF-8
% !TeX spellcheck = pt_BR
% !TeX root = ../template.tex
% !Tex tags = 

\question[10]

Crie um programa que receba três valores (obrigatoriamente maiores que zero), representando as medidas dos três lados de um triângulo. Elabore funções para:

\begin{parts}
	\part Determinar se eles lados formam um triângulo, sabendo que:
		\begin{itemize}
			\item O comprimento de cada lado de um triângulo é menor do que a soma dos outros dois lados.
		\end{itemize}
	\part Determinar e mostrar o tipo de triângulo, caso as medidas formem um triângulo. Sendo que:
		\begin{itemize}
			\item Chama-se \textbf{equilátero} o triângulo que tem três lados iguais.
			\item Denominam-se \textbf{isósceles} o triângulo que tem o comprimento de dois lados iguais.
			\item Recebe o nome de \textbf{escaleno} o triângulo que tem os três lados diferentes.
		\end{itemize}
\end{parts}