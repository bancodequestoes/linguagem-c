% !TeX encoding = UTF-8
% !TeX spellcheck = pt_BR
% !TeX root = ../template.tex
% !Tex tags = 

\question[10]

Escreva um programa em C que faça o jogo de ``adivinhar um número'' da forma que se segue: Seu programa escolhe um número para ser adivinhado selecionando um inteiro aleatoriamente no intervalo de 1 a 1000. O programa então escreve:

\begin{exemplo}{}{}
Tenho um numero entre 1 e 1000 \newline
Você pode adivinhar meu numero? \newline 
Por favor, digite seu primeiro palpite. 
\end{exemplo}

O jogador digita então o primeiro palpite. O programa escreve uma das respostas seguintes:

\begin{exemplo}{}{}
	Excelente! Você adivinhou o numero! Você gostaria de tentar novamente? \newline
	Muito baixo. Tente novamente. \newline
	Muito alto. Tente novamente.
\end{exemplo}

Se o palpite do jogador estiver incorreto, seu programa deve fazer um laço até que o jogador finalmente acerte o número. Seu programa deve continuar dizendo \textbf{Muito alto} ou \textbf{Muito baixo} para ajudar o jogador a ``chegar'' na resposta correta. Nota: A técnica de busca empregada neste problema é chamada \textit{pesquisa binaria}.

\begin{solution}
\begin{lstlisting}
#include <stdio.h>

int main(void) {

	int terminar = 0, numero = 43, palpite=-1;
	printf("Tenho um numero entre 1 e 1000.\n");
	printf("Voce pode adivinhar o meu numero?\n");
	
	do{
		do{
			printf("Por favor, digite o seu palpite:");
			scanf("%d", &palpite);
			
			if(palpite < numero){
				printf("Muito baixo! Tente novamente.\n");
			}
			if(palpite > numero){
				printf("Muito alto! Tente novamente.\n");
			}
		}while(palpite != numero);
		
		printf("Excelente! Voce adivinhou o numero!\n");
		printf("Voce gostaria de tentar novamente?");
		scanf("%d", &terminar);
	
	}while(terminar == 0);
	
	printf("Acabaou");
}
\end{lstlisting}
\end{solution}