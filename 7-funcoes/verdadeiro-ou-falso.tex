% !TeX encoding = UTF-8
% !TeX spellcheck = pt_BR
% !TeX root = ../template.tex
% !Tex tags = 

\question[10]

Use (\textbf{V}) para verdadeiro e (\textbf{F}) para falso nas seguintes sentenças.

\begin{parts}
	
	\part \tf[F]  Uma função pode ser declarada dentro de outra função.

	\part \tf[V]  Funções não podem retornar mais do que um valor por vez.

	\part \tf[V]  Se o tipo do retorno de uma função não é especificado, o tipo padrão é \emph{int}.

	\part \tf[F]  Em C, todas as funções, exceto a função \emph{main}, pode ser chamada recursivamente (uma função chama ela mesma).
	
	\part \tf[F] Funções devem ter no máximo um único comando \emph{return}.
	
	\part \tf[F] Palavras-chaves podem ser usadas como nomes de funções.
	
	\part \tf[V] Os argumentos em funções são opcionais.
	
	\part \tf[F] Toda função deve retornar um valor.

	\part \tf[F] O número máximo de argumentos que uma função pode ter é 12.

	\part \tf[V] É possível usar os comandos \emph{printf} e \emph{scanf} dentro de funções.

	\part \tf[V] O tipo de retorno \emph{void} deve ser usado quando uma função não retorna nada.

	\part \tf[F] É possível ter mais do que uma função \emph{main}.

	\part \tf[F] A função \emph{main} não é obrigatória.
	
	\part \tf[V] Variáveis declaradas em uma função não podem ser acessadas por outra função.
	
	\part \tf[F] Funções não podem retornar valores decimais.

	\part \tf[V] Uma função é construída com o intuito de realizar uma tarefa específica e bem-definida.
	
	\part \tf[V] De modo geral, os comandos \emph{printf} e \emph{scanf} devem ser evitados em funções.

	\part \tf[F] Não é obrigatório informar o tipo dos parâmetros em uma função.
	
	\part \tf[V] Quando o comando \emph{return} é executado, a função termina imediatamente.

	\part \tf[V] Uma função precisa ser declarada antes de ser utilizada.
	
	\part \tf[V] Em C, os parâmetros de uma função são sempre passados por \textbf{valor}.

	\part \tf[V] O C considera as funções \emph{soma} e \emph{Soma} idênticas.
						
\end{parts}
